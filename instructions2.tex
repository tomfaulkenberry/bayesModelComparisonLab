\documentclass[11pt]{article}
\usepackage[left=1in,right=1in,top=1in,bottom=1in]{geometry}
\usepackage{amsmath}
\usepackage{fancyhdr}
\lhead{Lab 2: Performing the model comparisons}
\rhead{PSYC 3435}
\pagestyle{fancy}
\setlength{\parindent}{0pt}
\setlength{\parskip}{2mm}

\begin{document}

\begin{enumerate}

\item Compute and report descriptives for demographic data, specifically mean and SD for age and frequency counts for genders.
  
\item Open the dataset in JASP and perform a Bayesian ANOVA on the mean RTs with format and problem size as fixed factors and subject number as a random factor. Construct an APA formatted table of descriptive statistics, showing at a minimum the mean RT and SD for each condition in the 2x2 design.

\item Construct a plot of mean RTs with format on the horizontal axis and problem size as two separate lines.  Do you think there is evidence for an interaction from these plots?  Explain.

\item JASP lists five models - describe each model in words.  Which model is the ``additive model''?  Which is the ``interactive model''?

\item Identify the posterior probabilities (i.e., $P(M\mid \text{data})$) for each model. Which model for receives the best support from the data?  What is the next best fitting model?

\item What does the column labeled $BF_M$ represent? What do these values tell you?
  
\item What does the column labeled $BF_{10}$ represent?  Use these values to compute the Bayes factor for the best model relative to the second best model. Explain what this Bayes factor tells you.

\item What do these results imply about the architecture of mental arithmetic?  
\end{enumerate}
% Emacs 25.3.1 (Org mode 8.2.10)
\end{document}