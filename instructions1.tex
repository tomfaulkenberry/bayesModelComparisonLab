\documentclass[11pt]{article}
\usepackage[left=1in,right=1in,top=1in,bottom=1in]{geometry}
\usepackage{fancyhdr}
\lhead{Lab 2: A mental arithmetic task}
\rhead{PSYC 3435}
\pagestyle{fancy}
\setlength{\parindent}{0pt}
\setlength{\parskip}{2mm}

\usepackage{array}
\newcolumntype{M}[1]{>{\centering\arraybackslash}m{#1}}
\newcolumntype{N}{@{}m{0pt}@{}}

\begin{document}

This experiment examines the effects of \emph{problem size} and \emph{format} on arithmetic performance.  The purpose of this document is to give you the instructions necessary for you to carry out the experiment.

\section*{Stimuli}
Attached are four problem sheets: two containing a list of addition problems written in Arabic digits, and two sheets of addition problems written in words.

\section*{Design}
The first independent variable is Format (digits vs. words).  This is called a \underline{between-subjects} manipulation, which means that each participant gets only one type of problem (either digits or words, but NOT both).  So, half of your participants will do the problems written in digits, and the other half will do the problems written in words.

The other independent variable is Problem Size (small vs. large); the two levels are denoted Form A and Form B, respectively.  This is called a \underline{within-subjects} manipulation, which means that each participant gets both small and large problems.  So, each participant will do both Form A and Form B.

Note that it is important to \emph{counterbalance} the orders of presentation in a within-subjects manipulation.  That means that half of your participants should do Form A then Form B, while the other half is reversed (Form B, then Form A).  

The data sheet in this packet should have enough information for you to keep track of who gets what and in what order!

\section*{Testing the participants}

\emph{Read carefully\ldots{}you must do the same thing for EVERY participant!}

\begin{itemize}
\item Make sure your participant is at least 18 years old.  Record their age and gender on the attached summary sheet.
\item Distribute the first of the four sheets of problems to the participant face down.  - Read the following instructions to the participant:
\begin{itemize}
\item ``This experiment is about mental arithmetic.  You will write the answers to as many problems as you can in 20 seconds.  Be quick, but accurate.  Write the answers in the provided boxes.  I will count down '3, 2, 1, start'.  When I say 'start', turn over the paper and start.  You may work the problems in any order. When I say 'stop', put your pencil down.''
\end{itemize}

\item Using a timer (a stopwatch or your phone), have the participant complete as many problems on the first sheet as possible in 20 seconds.  Do it the way you explained in the instruction statement (countdown 3, 2, 1, start).

\item After the participant completes sheet 1, distribute the second sheet to him/her face down.  Explain that he/she will do one more sheet of problems.

\item Using your timer, have the participant complete as many problems as possible in 20 seconds (just like the first sheet).

\item Continue this process until all four sheets have been completed.  Thank the participant for his/her participation.
\end{itemize}

\section*{Recording the data}

On a separate sheet of paper (see the provided data summary sheet), report the number of problems completed \emph{correctly} on each form as well as the number of errors on each form (note: there will be 8 numbers for each participant).  Turn in this summary sheet in class on the due date.

\newpage 

\section*{Data Collection Sheet}

Student researcher's name: \rule{3in}{0.5pt}\\

(this is your name, NOT the name of any of your participants)\\[5mm]

Count the \underline{number of problems completed} and the \underline{number of errors} on Form A and Form B for each participant.  Record these numbers in the appropriate cells:

\begin{table}[h!]
	\begin{center}
	\begin{tabular}{|M{1.5cm}|M{1cm}|M{1.2cm}|M{1.2cm}|M{1.2cm}|M{2cm}|M{1.2cm}|M{2cm}|M{1.2cm}|N}
		\hline
		& & & & & \multicolumn{2}{|c|}{Form A} & \multicolumn{2}{|c|}{Form B} &\\[5mm]
		\hline
		Subject & Age & Gender & Format & Order & Completed & Errors & Completed & Errors & \\[5mm]
		\hline
		1 & & & Digits & A -- B & & & & &\\[5mm]
		\hline
		2 & & & Digits & B -- A & & & & &\\[5mm]
		\hline
		3 & & & Words & A -- B & & & & &\\[5mm]
		\hline
		4 & & & Words & B -- A & & & & &\\[5mm]
		\hline
	
		
	\end{tabular}
	\end{center}
	
\end{table}

\vspace{1cm}

Return the completed data sheet in class on \textbf{Friday, February 13}.


\end{document}