\documentclass[11pt]{article}
\usepackage[left=1in,right=1in,top=1in,bottom=1in]{geometry}
\usepackage{amsmath}
\usepackage{fancyhdr}
\lhead{Lab 2: Writing the manuscript}
\rhead{PSYC 3435}
\pagestyle{fancy}
\setlength{\parindent}{0pt}
\setlength{\parskip}{2mm}

\begin{document}
As we did with Lab 1, your job now is to write up the experiment as an APA-formatted manscript. Here are the guidelines:

\begin{enumerate}
\item Your introduction section should contain the following three items:
\begin{itemize}
\item statement of problem (follow Campbell and Fugelsang; we are trying to determine whether encoding and calculation are functionally independent stages).
\item explain the two competing models. There are lots of ways to do this, but I like to couch it in terms of ``additive models'' (e.g., Dehaene \& Cohen, 1995) versus ``interactive models'' (e.g., Campbell, 1999).
\item describe our basic experiment (conceptual, not detailed) and lay out what the predictions of the two competing models would be.
\end{itemize}

\item Your Method section is intended to communicate with people who know very little about what was done. Be certain, however, that you include enough information so the study could be replicated.  If you are not sure of something (e.g., how the stimuli were chosen, etc.), ask me!  Also, looking at the original Campbell \& Fugelsang paper will help.

\item Pay particular attention to your results section. Please perform the following steps:  
\begin{itemize}
\item perform a Bayesian ANOVA in Jasp
\item define the models (there should be 5 of them)
\item explain how the data shifted the prior probabilities of the models into posterior probabilities.  In particular, which model was preferred after seeing the data?
\item what was the Bayes factor for this model in terms of overall model odds (i.e., the \(BF_M\) column)?
\item what was the Bayes factor for this winning model over the next best fitting model? Use the \(BF_{10}\) column to calculate this.
\end{itemize}
\end{enumerate}


\begin{enumerate}
\item Like all discussion sections, you should do the following
\begin{itemize}
\item briefly restate the purpose of your study
\item briefly describe your results
\item discuss your results in terms of competing models of mental arithmetic.
\item describe a followup study you could do.
\end{itemize}

\item Your reference section should include at least the following:
\begin{itemize}
\item Campbell \& Fugelsang (2001)
\item Dehaene \& Cohen (1995)
\item Campbell (1999)
\item at least one more paper related to these.
\end{itemize}
\end{enumerate}

Your completed manuscript will be submitted on Blackboard.
\end{document}